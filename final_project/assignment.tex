\documentclass{article}

\usepackage{amsmath}
\usepackage[utf8]{inputenc}
\usepackage[greek,english]{babel}
\usepackage{alphabeta}
\usepackage{graphicx}
\usepackage{lscape} 
\usepackage[title]{appendix}
\usepackage[greek,english]{babel}
\addto\captionsenglish{\renewcommand{\figurename}{Εικόνα}}
\addto\captionsenglish{\renewcommand{\contentsname}{Περιεχόμενα}}
\addto\captionsenglish{\renewcommand{\appendixname}{Παράρτημα}}
\graphicspath{ {./images/} }

 

\begin{document}
 
\title{Εργασία Υπολογιστικών Μαθηματικών 2019}
\author{
Χρήστος Γεωργίου Μουσσές\\A.M: 4206\\\\
Σωτηρία Δουλγεράκη\\A.M: 3218\\\\
Πασχάλης Πιπίδης\\A.M: 3064
}
\date{}

\maketitle
\includegraphics[width = 1\textwidth]{robot}

\newpage
\tableofcontents
\newpage


\section{Σύστημα Ρομποτικής Πλατφόρμας}

Το σύστημά μας περιγράφεται από τις παρακάτω τρεις διαφορικές εξισώσεις δεύτερης τάξης: 

\begin{equation}
(m + 3m_a)(cos(\psi)\ddot{x} - sin(\psi)\dot{\psi}\dot{x} + cos(\psi)\dot{\psi}\dot{y}) = f_x - D_x|\dot{x}|\dot{x}
\end{equation}

\begin{equation}
(m + 3m_a)(cos(\psi)\ddot{y} - cos(\psi)\dot{\psi}\dot{x} - sin(\psi)\dot{\psi}\dot{y}) = f_y - D_y|\dot{y}|\dot{y}
\end{equation}

\begin{equation}
m_z\ddot{\psi} = n_z - D_\psi|\dot{\psi}|\dot{\psi}
\end{equation}

\section{Μέθοδοι Ολοκλήρωσης}\
Για να χρησιμοποιηθεί η μέθοδος Euler και η τροποποιημένη μέθοδος Euler, απαιτούνται διαφορικές εξισώσεις πρώτης τάξης, οπότε χρησιμοποιούνται νέες μεταβλητές, όπου:

\[
x_1 = x
\]
\[
y_1 = y
\]
\[
\psi_1 = \psi
\]
και
\[
x_2 = \dot{x_1} = \dot{x} 
\]
\[
y_2 = \dot{y_1} = \dot{y} 
\]
\[
\psi_2 = \dot{\psi_1} = \dot{\psi} 
\]
\\\\\\\\
Με τις νέες μεταβλητές, η εξίσωση (1) μετατρέπεται στο παρακάτω σύστημα δύο διαφορικών εξισώσεων πρώτης τάξης:

\begin{equation}
\dot{x_1} = x_2
\end{equation}

\begin{equation}
\dot{x_2} = tan(\psi_1)\psi_2x_2 - \psi_2y_2 + \frac{f_x - D_x|x_2|x_2}{(m + 3m_a)cos(\psi_1)} 
\end{equation}
\\
Αντίστοιχα η (2) και η (3) μετατρέπονται σε:

\begin{equation}
\dot{y_1} = y_2
\end{equation}

\begin{equation}
\dot{y_2} = tan(\psi_1)\psi_2y_2 + \psi_2x_2 + \frac{f_y - D_y|y_2|y_2}{(m + 3m_a)cos(\psi_1)} 
\end{equation}
\\
και
\\
\begin{equation}
\dot{\psi_1} = \psi_2
\end{equation}

\begin{equation}
\dot{\psi_2} = \frac{n_z - D_\psi|\psi_2|\psi_2}{m_z} 
\end{equation}

\subsection{Μέθοδος Euler}
Στη μέθοδο Euler, η λύση του συστήματος πρώτης τάξης \(\dot{y}(t) = f(t, y(t))\) με αρχική συνθήκη \(y(0)\), στις χρονικές στιγμές \(t = kh, k = 0, 1, 2...\) προσεγγίζεται από τη σχέση \(y(k+1) = y(k) + hf(hk, y(k))\) όπου \(y(k) = y(t = kh), k = 0, 1, 2...\) και \(h\) το βήμα ολοκλήρωσης.
\\\\
Για το σύστημα των εξισώσεων (4) έως (9), η εφαρμογή της παραπάνω σχέσης δίνει:

\[x_1(k+1) = x_1(k) + hx_2(k)\]
\[x_2(k+1) = x_2(k) + h[tan(\psi_1(k))\psi_2(k)x_2(k) - \psi_2(k)y_2(k) + \frac{f_x - D_x|x_2(k)|x_2(k)}{(m + 3m_a)cos(\psi_1(k))}]\]

\[ y_1(k+1)   = y_1(k) + hy_2(k)   \]
\[y_2(k+1) = y_2(k) + h[tan(\psi_1(k))\psi_2(k)y_2(k) + \psi_2(k)x_2(k) + \frac{f_y - D_y|y_2(k)|y_2(k)}{(m + 3m_a)cos(\psi_1(k))}]\]


\[ \psi_1(k+1)   = \psi_1(k) + h\psi_2(k)   \]
\[ \psi_2(k+1) = \psi_2(k) + h[\frac{n_z - D_\psi|\psi_2(k)|\psi_2(k)}{m_z}]\]

όπου \(f_x, f_y, n_z\) είναι σταθερά.
\subsection{Τροποποιημένη Μέθοδος Euler}
Στην τροποποιημένη μέθοδο Euler, η λύση του συστήματος πρώτης τάξης \(\dot{y}(t) = f(t, y(t))\) με αρχική συνθήκη \(y(0)\), στις χρονικές στιγμές \(t = kh, k = 0, 1, 2...\) προσεγγίζεται από τη σχέση \(y(k+1) = y(k) + hf(kh + \frac{h}{2}, y(k) + \frac{h}{2}f(kh, y(k)))\) όπου \(y(k) = y(t = kh), k = 0, 1, 2...\) και \(h\) το βήμα ολοκλήρωσης.
\\\\
Για το σύστημα των εξισώσεων (4) έως (9), η εφαρμογή της παραπάνω σχέσης γίνεται σε δύο βήματα:
\\\\
Πρώτα υπολογίζεται το τμήμα \(y(k) + \frac{h}{2}f(kh, y(k))\) από τις σχέσεις: 
\[g_1(k) = x_1(k) + \frac{h}{2}x_2(k)\]
\[g_2(k) = x_2(k) + \frac{h}{2}[tan(\psi_1(k))\psi_2(k)x_2(k) - \psi_2(k)y_2(k) + \frac{f_x - D_x|x_2(k)|x_2(k)}{(m + 3m_a)cos(\psi_1(k))}]\]

\[ g_3(k)   = y_1(k) + \frac{h}{2}y_2(k)   \]
\[g_4(k) = y_2(k) + \frac{h}{2}[tan(\psi_1(k))\psi_2(k)y_2(k) + \psi_2(k)x_2(k) + \frac{f_y - D_y|y_2(k)|y_2(k)}{(m + 3m_a)cos(\psi_1(k))}]\]


\[ g_5(k)   = \psi_1(k) + \frac{h}{2}\psi_2(k)   \]
\[ g_6(k) = \psi_2(k) + \frac{h}{2}[\frac{n_z - D_\psi|\psi_2(k)|\psi_2(k)}{m_z}]\]
\\
και στη συνέχεια το \(y(k) + hf(kh + \frac{h}{2}, y(k) + \frac{h}{2}f(kh, y(k)))\) υπολογίζεται από τις σχέσεις:

\[x_1(k+1) = x_1(k) + hg_2(k)\]
\[x_2(k+1) = x_2(k) + h[tan(g_5(k))g_6(k)g_2(k) - g_6(k)g_4(k) + \frac{f_x - D_x|g_2(k)|g_2(k)}{(m + 3m_a)cos(g_5(k))}]\]

\[ y_1(k+1)   = y_1(k) + hg_4(k)   \]
\[y_2(k+1) = y_2(k) + h[tan(g_5(k))g_6(k)g_4(k) + g_6(k)g_2(k) + \frac{f_y - D_y|g_4(k)|g_4(k)}{(m + 3m_a)cos(g_5(k))}]\]


\[ \psi_1(k+1)   = \psi_1(k) + hg_6(k)   \]
\[ \psi_2(k+1) = \psi_2(k) + h[\frac{n_z - D_\psi|g_6(k)|g_6(k)}{m_z}]\]
\\
βάζοντας τα \(g_1(k), g_2(k), g_3(k), g_4(k), g_5(k), g_6(k)\) στη θέση των \(x_1(k), x_2(k), y_1(k), y_2(k), \psi_1(k), \psi_2(k)\), αντίστοιχα.

\subsection{Προβλήματα της Προσομοίωσης σε Υπολογιστή και Αντιμετώπισή τους}
Επιλέχθηκε η υλοποίηση των μεθόδων Euler να γίνει σε γλώσσα C και κατά τη συγγραφή του κώδικα χρησιμοποιήθηκε αρχικά αναπαράσταση των μεταβλητών με τον τύπο long double ακρίβειας 80 bits. Κατά τη σύγκριση όμως των αποτελεσμάτων της υλοποίησής μας με τα αποτελέσματα που πήραμε από το MATLAB, το οποίο χρησιμοποιεί πιο ακριβείς μεθόδους, προέκυψαν μεγάλες διαφορές. Προσπαθώντας να επιλύσουμε το πρόβλημα της μεγάλης διαφοράς, οδηγηθήκαμε στο συμπέρασμα ότι πρέπει να χρησιμοποιήσουμε αναπαράσταση με μεγαλύτερη ακρίβεια και πιο συγκεκριμένα χρησιμοποιήθηκε η βιβλιοθήκη quadmath.h, η οποία επιτρέπει αναπαράσταση με 128 bits. Όλες οι προσομοιώσεις λοιπόν που ακολουθούν χρησιμοποιούν αναπαράσταση με 128 bits για μεγαλύτερη ακρίβεια και η αιτιολόγηση της επιλογής εξηγείται στο παράρτημα.

\subsection{Προσoμοίωση}
Προσομοιώθηκε το παραπάνω σύστημα εξισώσεων σε γλώσσα προγραμματισμού C και έγινε εξαγωγή των αποτελεσμάτων σε αρχεία ASCII τα οποία στη συνέχεια εισήχθησαν στο MATLAB για την απεικόνιση των γραφικών παραστάσεων.
\\\\
Οι γραφικές παραστάσεις που παρατίθενται στη συνέχεια, αφορούν τις τρείς παρακάτω περιπτώσεις τιμών για τις εισόδους του συστήματος:
\[[f_x, f_y, n_z]^T = [3496, 0, 0]^T\]
\[[f_x, f_y, n_z]^T = [0, -3496, 0]^T\]
\[[f_x, f_y, n_z]^T = [0, 0, -3496]^T\] 
 \\
Και στις τρείς περιπτώσεις οι αρχικές συνθήκες που χρησιμοποιήθηκαν είναι ίδιες:
\[x(0) = 0.3496\]
\[\dot{x}(0) = 0\]
\[y(0) = 0\]
\[\dot{y}(0) = 0\]
\[\psi(0) = 0\]
\[\dot{\psi}(0) = 0\]
Οι παράμετροι του συστήματος υπολογίστηκαν με \(Α.Μ = 3496\)\footnote{\[A.M = \frac{4206 + 3218 + 3064}{3} = 3496\]} και δίνονται παρακάτω:
\[D_x = 11835\]
\[D_y = 8339\]
\[D_\psi = 15331\]
\[m = 425000\]
\[m_z = 357000000\]
\[m_\alpha = 113000\]

\newpage
\subsection{Γραφικές Παραστάσεις}
Σε καθεμία από τις περιπτώσεις παρουσιάζονται μόνο οι μη μηδενικές καταστάσεις του συστήματος ενώ οι υπόλοιπες παραμένουν μηδέν σε όλη τη διάρκεια της προσομοίωσης γιατί οι αντίστοιχες είσοδοι είναι μηδέν όπως και οι αρχικές συνθήκες.
\\\\
Επειδή στα συγκριτικά γραφήματα Euler - Τροποποιημένη Euler που ακολουθούν, οι διαφορές είναι πολύ μικρές και επομένως μη ορατές, έχουν προστεθεί και μεγεθύνσεις στις οποίες φαίνονται οι διαφορές.
\subsubsection{Περίπτωση Μεταβολής μόνο της εισόδου \(f_x\) (Εικόνες 1-4)}
Παρουσιάζεται η μεταβολή των \(x_1(t)\) και \(x_2(t)\) με το χρόνο. Παρουσιάζεται στις εικόνες \ref{l_2} και \ref{l_4} μεγέθυνση σε μια περιοχή. 

\subsubsection{Περίπτωση Μεταβολής μόνο της εισόδου \(f_y\) (Εικόνες 5-8)}
Παρουσιάζεται η μεταβολή των \(y_1(t)\) και \(y_2(t)\) με το χρόνο. Παρουσιάζεται στις εικόνες \ref{l_6} και \ref{l_8} μεγέθυνση σε μια περιοχή.

\subsubsection{Περίπτωση Μεταβολής μόνο της εισόδου \(n_z\) (Εικόνες 9-12)}
Παρουσιάζεται η μεταβολή των \(\psi_1(t)\) και \(\psi_2(t)\) με το χρόνο. Παρουσιάζεται στις εικόνες \ref{l_10} και \ref{l_12} μεγέθυνση σε μια περιοχή.
\clearpage

\begin{figure}[bh!]
 \centering
\includegraphics[scale = 0.7]{fxEuler_x1}
\caption{Μεταβολή της $x_1(t)$ για Euler και τροποποιημένη
 Euler (οι διαφορές είναι μικρές)}
\end{figure}

\begin{figure}[h!]
 \centering
\includegraphics[scale = 0.7]{fxEuler_x1_zoom}
\caption{Μεταβολή της $x_1(t)$ για Euler και τροποποιημένη Euler (μεγέθυνση)}
\label{l_2}
\end{figure}
\clearpage

\clearpage
\begin{figure}[h!]
 \centering
\includegraphics[scale = 0.7]{fxEuler_x2}
\caption{Μεταβολή της $x_2(t)$ για Euler και τροποποιημένη Euler (οι διαφορές είναι μικρές)}
\end{figure}


\begin{figure}[h!]
 \centering
\includegraphics[scale = 0.7]{fxEuler_x2_zoom}
\caption{Μεταβολή της $x_2(t)$ για Euler και τροποποιημένη Euler (μεγέθυνση)}
\label{l_4}
\end{figure}
\clearpage

\clearpage
\begin{figure}[h!]
 \centering
\includegraphics[scale = 0.7]{fyEuler_y1}
\caption{Μεταβολή της $y_1(t)$ για Euler και τροποποιημένη
 Euler (οι διαφορές είναι μικρές)}
\end{figure}

\begin{figure}[h!]
 \centering
\includegraphics[scale = 0.7]{fyEuler_y1_zoom}
\caption{Μεταβολή της $y_1(t)$ για Euler και τροποποιημένη Euler (μεγέθυνση)}
\label{l_6}
\end{figure}
\clearpage

\clearpage
\begin{figure}[h!]
\centering
\includegraphics[scale = 0.7]{fyEuler_y2}
\caption{Μεταβολή της $y_2(t)$ για Euler και τροποποιημένη Euler (οι διαφορές είναι μικρές)}
\end{figure}


\begin{figure}[h!]
 \centering
\includegraphics[scale = 0.7]{fyEuler_y2_zoom}
\caption{Μεταβολή της $y_2(t)$ για Euler και τροποποιημένη Euler (μεγέθυνση)}
\label{l_8}
\end{figure}
\clearpage

\clearpage
\begin{figure}[h!]
\includegraphics[scale = 0.7]{nzEuler_c1}
\caption{Μεταβολή της $\psi_1(t)$ για Euler και τροποποιημένη
 Euler (οι διαφορές είναι μικρές)}
\end{figure}

\begin{figure}[h!]
\includegraphics[scale = 0.7]{nzEuler_c1_zoom}
\caption{Μεταβολή της $\psi_1(t)$ για Euler και τροποποιημένη Euler (μεγέθυνση)}
\label{l_10}
\end{figure}
\clearpage

\clearpage
\begin{figure}[h!]
\includegraphics[scale = 0.5]{nzEuler_c2}
\caption{Μεταβολή της $\psi_2(t)$ για Euler και τροποποιημένη Euler (οι διαφορές είναι μικρές)}
\end{figure}

\begin{figure}[h!]
\includegraphics[scale = 0.5]{nzEuler_c2_zoom}
\caption{Μεταβολή της $\psi_2(t)$ για Euler και τροποποιημένη Euler (μεγέθυνση)}
\label{l_12}
\end{figure}
\clearpage

\section{Σύστημα αυτομάτου ελέγχου κλειστού βρόγχου}
Χρησιμοποιώντας για τις εισόδους τους παρακάτω ελεγκτές αναλογικού διαφορικού τύπου στις εξισώσεις (4)-(9) :

\[f_x = K_{px}(x_{des} - x) - K_{dx}\dot{x} \]
\[f_y = K_{py}(y_{des} - y) - K_{dy}\dot{y} \]
\[n_z = K_{p\psi}(\psi_{des} - \psi) - K_{d\psi}\dot{\psi} \]
\\
και αναδιατυπώνοντας, χρησιμοποιώντας τις μεταβλητές \(x_1, x_2, y_1, y_2, \psi_1, \psi_2\) όπως και πριν, καταλήγουμε στο παρακάτω σύστημα κλειστού βρόγχου διαφορικών εξισώσεων πρώτης τάξης:



\begin{equation}
\dot{x_1} = x_2
\end{equation}

\begin{equation}
\dot{x_2} = tan(\psi_1)\psi_2x_2 - \psi_2y_2 + \frac{K_{px}(x_{des} - x_1) - K_{dx}x_2 - D_x|x_2|x_2}{(m + 3m_a)cos(\psi_1)} 
\end{equation}

\begin{equation}
\dot{y_1} = y_2
\end{equation}

\begin{equation}
\dot{y_2} = tan(\psi_1)\psi_2y_2 + \psi_2x_2 + \frac{K_{py}(y_{des} - y_1) - K_{dy}y_2 - D_y|y_2|y_2}{(m + 3m_a)cos(\psi_1)} 
\end{equation}

\begin{equation}
\dot{\psi_1} = \psi_2
\end{equation}

\begin{equation}
\dot{\psi_2} = \frac{K_{p\psi}(\psi_{des} - \psi_1) - K_{d\psi}\psi_2  - D_\psi|\psi_2|\psi_2}{m_z} 
\end{equation}

\subsection{Προσομοίωση}
Όπως και πριν, χρησιμοποιώντας τον ίδιο κώδικα σε C, όπου ενσωματώθηκαν οι αλλαγές των εξισώσεων, προσομοιώθηκαν οι μέθοδοι Euler και τροποποιημένη Euler. Oι αρχικές συνθήκες που χρησιμοποιήθηκαν είναι:
\[x(0) = 0\]
\[\dot{x}(0) = 0\]
\[y(0) = -3,496\]
\[\dot{y}(0) = 0\]
\[\psi(0) = 0\]
\[\dot{\psi}(0) = 0\]
Οι παράμετροι του συστήματος και οι παράμετροι του ελεγκτή δίνονται παρακάτω:
\[D_x = 15331\]
\[D_y = 11835\]
\[D_\psi = 11835\]
\[m = 425000\]
\[m_z = 357000000\]
\[m_\alpha = 113000\]
\[K_{px} = 77480\]
\[K_{dx} = 5000000\]
\[K_{py} = 60000\]
\[K_{dy} = 4650400\]
\[K_{p\psi} = 50000\]
\[K_{d\psi} = 7000000\]
\\
Οι επιθυμητές τελικές τιμές για τα \(x, y, \psi\) είναι αντίστοιχα: 

\[x_{des} = 34.96\]
\[y_{des} = -34.96\]
\[\psi_{des} = 0.3496\]
\\
Έπειτα έγινε απεικόνιση των γραφικών παραστάσεων σε MATLAB.

\subsection{Γραφικές Παραστάσεις}
Δίνονται στη συνέχεια, οι μεταβολές όλων των καταστάσεων απ' όπου βλέπουμε οτι πράγματι τα \(x, y, \psi\) τείνουν ασυμπτωτικά στις επιθυμητές τιμές και για τις δύο μεθόδους επίλυσης. 
\\\\
Επειδή στα συγκριτικά γραφήματα Euler - Τροποποιημένη Euler που ακολουθούν, οι διαφορές είναι πολύ μικρές και επομένως μη ορατές, έχουν προστεθεί και μεγεθύνσεις στις οποίες φαίνονται οι διαφορές. 

\clearpage

\begin{figure}[bh!]
 \centering
\includegraphics[scale = 0.7]{clEuler_x1}
\caption{Μεταβολή της $x_1(t)$ για Euler και τροποποιημένη
 Euler}
\end{figure}

\begin{figure}[h!]
 \centering
\includegraphics[scale = 0.7]{clEuler_x1_zoom}
\caption{Μεταβολή της $x_1(t)$ για Euler και τροποποιημένη Euler (μεγέθυνση)}
\end{figure}
\clearpage

\begin{figure}[bh!]
 \centering
\includegraphics[scale = 0.7]{clEuler_x2}
\caption{Μεταβολή της $x_2(t)$ για Euler και τροποποιημένη
 Euler}
\end{figure}

\begin{figure}[h!]
 \centering
\includegraphics[scale = 0.7]{clEuler_x2_zoom}
\caption{Μεταβολή της $x_2(t)$ για Euler και τροποποιημένη Euler (μεγέθυνση)}
\end{figure}
\clearpage

\begin{figure}[bh!]
 \centering
\includegraphics[scale = 0.7]{clEuler_y1}
\caption{Μεταβολή της $y_1(t)$ για Euler και τροποποιημένη
 Euler}
\end{figure}

\begin{figure}[h!]
 \centering
\includegraphics[scale = 0.7]{clEuler_y1_zoom}
\caption{Μεταβολή της $y_1(t)$ για Euler και τροποποιημένη Euler (μεγέθυνση)}
\end{figure}
\clearpage

\begin{figure}[bh!]
 \centering
\includegraphics[scale = 0.7]{clEuler_y2}
\caption{Μεταβολή της $y_2(t)$ για Euler και τροποποιημένη
 Euler}
\end{figure}

\begin{figure}[h!]
 \centering
\includegraphics[scale = 0.7]{clEuler_y2_zoom}
\caption{Μεταβολή της $y_2(t)$ για Euler και τροποποιημένη Euler (μεγέθυνση)}
\end{figure}
\clearpage

\begin{figure}[bh!]
 \centering
\includegraphics[scale = 0.7]{clEuler_c1}
\caption{Μεταβολή της $\psi_1(t)$ για Euler και τροποποιημένη
 Euler}
\end{figure}

\begin{figure}[h!]
 \centering
\includegraphics[scale = 0.7]{clEuler_c1_zoom}
\caption{Μεταβολή της $\psi_1(t)$ για Euler και τροποποιημένη Euler (μεγέθυνση)}
\end{figure}
\clearpage

\begin{figure}[bh!]
 \centering
\includegraphics[scale = 0.5]{clEuler_c2}
\caption{Μεταβολή της $\psi_2(t)$ για Euler και τροποποιημένη
 Euler}
\end{figure}

\begin{figure}[h!]
 \centering
\includegraphics[scale = 0.5]{clEuler_c2_zoom}
\caption{Μεταβολή της $\psi_2(t)$ για Euler και τροποποιημένη Euler (μεγέθυνση)}
\end{figure}
\clearpage

\section{Γραμμικό Σύστημα}
Προσεγγίζοντάς την (3) με τη γραμμική διαφορική εξίσωση 
\begin{equation}
m_z\ddot{\psi} = K_{p\psi}(\psi_{des} - \psi) - K_{d\psi}\dot{\psi} - D_\psi\dot{\psi}
\end{equation}
απλοποιείται η μελέτη της περιστροφής του ρομπότ.
\subsection{Συνάρτηση Μεταφοράς}
Στη γραμμική εξίσωση (16) εφαρμόζουμε μετασχηματισμό Laplace και στα δύο μέρη, με μηδενικές αρχικές συνθήκες. Έχουμε:
\[m_zs^2Ψ(s) = K_{p\psi}(Ψ_{des}(s) - Ψ(s)) - K_{d\psi}sΨ(s) - D_\psi s Ψ(s)\]
\[\Rightarrow m_zs^2Ψ(s) + (K_{d\psi} + D_\psi)sΨ(s) + K_{p\psi}Ψ(s) = K_{p\psi}Ψ_{des}(s)\]
οπότε η συνάρτηση μεταφοράς ανάμεσα στην επιθυμητή τιμή \(Ψ_{des}(s)\) και την έξοδο \(Ψ(s)\) είναι:
\[H(s) = \frac{Ψ(s)}{Ψ_{des}(s)} = \frac{K_{p\psi}}{m_zs^2 + (K_{d\psi} + D_\psi)s + K_{p\psi}}\]
\[= \frac{1}{\frac{m_z}{K_{p\psi}}s^2 + \frac{(K_{d\psi} + D\psi)}{K_{p\psi}}s +1}\]
Η συνάρτηση έχει μόνο πόλους, τις ρίζες της 
\begin{equation}
\frac{m_z}{K_{p\psi}}s^2 + \frac{(K_{d\psi} + D\psi)}{K_{p\psi}}s +1 = 0
\end{equation}
\\Με αντικατάσταση τιμών η εξίσωση γίνεται \(7140s^2 + 140.2367s + 1 = 0\) με δύο συζυγείς μιγαδικούς πόλους \(a\pm bj = -0.0098204972 \pm 0.0066040788j\).
\subsection{Γεωμετρικός Τόπος Ριζών}
Για διάφορες τιμές των \(K_{p\psi}, K_{d\psi}\) υπολογίσαμε, μέσω του MATLAB, τις ρίζες της εξίσωσης (17), όπου \(m_z = 357000000\) και \(D_\psi = 11835\), δηλαδή τους πόλους της συνάρτησης μεταφοράς του γραμμικού συστήματος. Χρησιμοποιήθηκαν οι τιμές:\\\\ \(10^{-15}, 10^{-5}, 0.001, 0.005, 0.01, 0.05, 0.1, 0.2, \\0.5, 1, 2, 5, 10, 20, 50, 100, 500, 1000, 10^5, 10^{15}\)
\\\\για το \(K_{p\psi}\), ενώ για κάθε τιμή του \(K_{p\psi}\) δίνουμε στο \(K_{d\psi}\) 200 διαφορετικές τιμές λογαριθμικά κατανεμημένες στο διάστημα \([10^{-5}, 10^{5}]\).
\\\\Προγραμματίζοντας στο MATLAB με χρήση της συνάρτησης roots() που βρίσκει τις ρίζες ενός πολυωνύμου, προέκυψε το γράφημα της εικόνας \ref{l_25}.
\\\\Σε καθένα από τα υπογραφήματα, για κάθε διαφορετική τιμή του \(K_{p\psi}\), δίνεται η μεταβολή των δύο πόλων του συστήματος (με μπλε και κόκκινο χρώμα αντίστοιχα) όταν το \(K_{d\psi}\) πάρει τις 200 διαφορετικές τιμές.
\\\\Για όλες τις μεταβολές των \(K_{p\psi}\), \(K_{d\psi}\), οι πόλοι παραμένουν στο αριστερό μιγαδικό επίπεδο(με αρνητικό πραγματικό μέρος) κι επομένως βγάζουμε το συμπέρασμα ότι το σύστημα είναι πάντα ευσταθές.
\\\\Τυπική εικόνα μεταβολής των πόλων είναι αυτή του γραφήματος για \(K_{p\psi} = 0.5\). Για μικρές τιμές του \(K_{d\psi}\), έχουμε δύο συζυγείς μιγαδικούς πόλους με μικρό αρνητικό πραγματικό μέρος (οπότε το σύστημα έχει σημαντικές ταλαντώσεις που μειώνονται). \\\\Όσο μεγαλώνει το \(K_{p\psi}\), τόσο αυτοί οι πόλοι έχουν μεγαλύτερο φανταστικό μέρος για την ίδια τιμή του \(K_{d\psi}\). Για σταθερό \(K_{p\psi}\), καθώς αυξάνεται το \(K_{d\psi}\), οι πόλοι κινούνται προς τον πραγματικό άξονα και για κάποια τιμή του \(K_{d\psi}\) έχουμε δύο ίδιους πραγματικούς πόλους, οπότε το σύστημα έχει στην απόκρισή του όρους \(e^{-at}\) και \(te^{-at}\) με \(a\) το διπλό πραγματικό πόλο. \\\\Για ακόμα μεγαλύτερες τιμές του \(K_{d\psi}\) έχουμε δύο πραγματικούς, διαφορετικούς μεταξύ τους πόλους \(a_1\) και \(a_2\) από τους οποίους ο ένας τείνει στο \(-\infty\) και ο άλλος στο 0, οπότε η απόκριση του συστήματος έχει όρους της μορφής \(e^{-a_1t}\) και \(e^{-a_2t}\).\\\\Επειδή κρατάμε σταθερή την περιοχή τιμών για το \(K_{d\psi}\) (στο διάστημα \([10^{-5}, 10^{5}]\)) για κάθε διαφορετική τιμή του \(K_{p\psi}\), για μικρότερες και μεγαλύτερες από το 0.5 τιμές του  \(K_{p\psi}\), βλέπουμε μόνο ένα τμήμα της παραπάνω καμπύλης.

\clearpage
\begin{landscape}                                                                             
\begin{figure}[hb]
 \centering
\includegraphics[scale = 0.49]{roots}
\caption{Μεταβολή των πόλων του γραμμικού συστήματος για διάφορες τιμές των \(K_{d\psi}\) και \(K_{d\psi}\) }
\label{l_25}
\end{figure}
\end{landscape}                                                                             
 \clearpage

\subsection{Αναλυτική λύση}
Η γραμμική διαφορική εξίσωση  γράφεται: 
\[\frac{m_z}{K_{p\psi}}\ddot{\psi}(t) + \frac{K_{d\psi} + D_\psi}{K_{p\psi}}\dot{\psi}(t) + \psi(t) = \psi_{des}\], με σταθερό επιθυμητό στόχο \(\psi_{des} = 0.3496\) και είναι μη ομογενής. Η αντίστοιχη ομογενής είναι: \[\frac{m_z}{K_{p\psi}}\ddot{\psi}(t) + \frac{K_{d\psi} + D_\psi}{K_{p\psi}}\dot{\psi}(t) + \psi(t) = 0\]έχει χαρακτηριστική εξίσωση τη (17) η οποία έχει δύο συζυγείς μιγαδικές ρίζες στο αριστερό μιγαδικό επίπεδο, τις \(a \pm bj\), με \(a < 0\).\\\\Η γενική λύση της ομογενούς είναι επομένως 
\[\psi_{ομ}(t) = c_1e^{at}sin(bt) + c_2e^{at}cos(bt)\]
με \(c_1, c_2\) σταθερές.
\\\\Επίσης είναι εύκολο να δούμε ότι η μη ομογενής διαφορική εξίσωση έχει ως μερική λύση τη  σταθερά \(\psi_{μερ}(t) = \psi_{des} = 0.3496\) για την οποία \(\dot{\psi}_{des} = \ddot{\psi}_{des} = 0\).
\\\\Έτσι τελικά η γενική λύση της μη ομογενούς γραμμικής διαφορικής εξίσωσης είναι η \(\psi(t) = \psi_{ομ}(t) + \psi_{μερ}(t) = \psi_{des} + c_1e^{at}sin(bt) + c_2e^{at}cos(bt)\)
\\\\Η παράγωγός της εύκολα υπολογίζεται σε
\[\dot{\psi}(t) = e^{at}[(c_1a - c_2b)sin(bt) + (c_1b + c_2a)cos(bt)]\]
\\Θα βρούμε τα \(c_1, c_2\) από τις αρχικές συνθήκες \(\psi(0) = 0\) και \(\dot{\psi}(0) = 0\).
\\\\Είναι \[\psi(0) =  0 = ac_1 + bc_2\] και \[\dot{\psi}(0) = 0 = c2 + \psi_{des}\]απ' όπου προκύπτει \[c_1 = \frac{b\psi_{des}}{a}\] και \[c_2 = -\psi_{des}\]
\\Τελικά η αναλυτική λύση είναι:
\begin{equation}
\psi(t) = \psi_{des} + \frac{b\psi_{des}}{a}e^{at}sin(bt) - \psi_{des}e^{at}cos(bt)
\end{equation}
και
\begin{equation}
\dot{\psi}(t) = - \frac{(a^2 + b^2)}{a}\psi_{des}e^{at}sin(bt)
\end{equation}
Αντικαθιστώντας τιμές έχουμε:
\[\psi(t) = 0.3496 - 0.2351e^{-0.0098t}sin(0.0066t) - 0.3496e^{-0.0098t}cos(0.0066t)\]
\[\dot{\psi}(t) = 0.0049e^{-0.0098t}sin(0.0066t)\]
\\Η λύση είναι ευσταθής, αφού το \(a = -0.0098 < 0\) και τα εκθετικά φθίνουν, με αποτέλεσμα όταν \(t \rightarrow \infty, \psi(t) \rightarrow \psi_{des} = 0.3496\) και \(\dot{\psi}(t) \rightarrow 0\)
\subsection{Προσομοίωση}
Αναδιατυπώνοντας πάλι τη διαφορική εξίσωση δεύτερης τάξης (16) ως δύο διαφορικές εξισώσεις πρώτης τάξης με μεταβλητές \(\psi_1, \psi_2\) καταλήγουμε στο σύστημα:
\begin{equation}
\dot{\psi_1} = \psi_2
\end{equation}

\begin{equation}
\dot{\psi_2} = \frac{K_{p\psi}(\psi_{des} - \psi_1) - (K_{d\psi} + D_\psi)\psi_2}{m_z}
\end{equation}
\\
Όπως και πριν, χρησιμοποιώντας τον ίδιο κώδικα σε C, τροποποιόντας τον κατάλληλα, προσομοιώθηκαν οι μέθοδοι Euler και τροποποιημένη Euler για να προσεγγιστεί η λύση της (16). Έπειτα γράφηκε κώδικας όπου υλοποιεί την αναλυτική της λύση.   
\\\\Oι αρχικές συνθήκες που χρησιμοποιήθηκαν είναι:
\[\psi(0) = 0\]
\[\dot{\psi}(0) = 0\]
Οι παράμετροι του συστήματος δίνονται παρακάτω:
\[D_\psi = 11835\]
\[m_z = 357000000\]
\[K_{p\psi} = 50000\]
\[K_{d\psi} = 7000000\]

Οι επιθυμητή τελική τιμή για το \(\psi\) είναι αντίστοιχα: 
\[\psi_{des} = 0.3496\]
Έπειτα έγινε πάλι απεικόνιση των γραφικών παραστάσεων σε MATLAB.
\subsection{Γραφικές Παραστάσεις}
Δίνονται στη συνέχεια, οι μεταβολές όλων των καταστάσεων, απ' όπου βλέπουμε ότι πράγματι τα \(\psi\) και \(\dot{\psi}\) τείνει ασυμπτωτικά στην επιθυμητή τιμή και για τις δύο μεθόδους επίλυσης. Επίσης, απεικονίζεται και η αναλυτική λύση με την οποία γίνεται και η σύγκριση των μεθόδων Euler.
\\\\Όπως ήταν και αναμενόμενο και όπως φαίνεται και από τις μεγεθύσεις που παρατίθενται, η τροποποιημένη μέθοδος Euler προσεγγίζει με μικρότερο σφάλμα την αναλυτική λύση από την απλή μέθοδο Euler.
\\\\
Επειδή στα συγκριτικά γραφήματα Euler - Tροποποιημένη Euler - Αναλυτική Λύση που ακολουθούν, οι διαφορές είναι πολύ μικρές και επομένως μη ορατές, έχουν προστεθεί και μεγεθύνσεις στις οποίες φαίνονται οι διαφορές.

\clearpage

\begin{figure}[bh!]
 \centering
\includegraphics[scale = 0.7]{lEuler_c1}
\caption{Μεταβολή της $\psi_1(t)$ για Euler, τροποποιημένη
 Euler και αναλυτική λύση (μικρές διαφορές)}
\end{figure}
 \clearpage
 
\begin{figure}[h!]
 \centering
\includegraphics[scale = 0.45]{lEuler_c1_zoom1}
\caption{Μεταβολή της $\psi_1(t)$ για Euler και αναλυτική λύση (μεγέθυνση)}
\end{figure}

\begin{figure}[h!]
 \centering
\includegraphics[scale = 0.45]{lEuler_c1_zoom2}
\caption{Μεταβολή της $\psi_1(t)$ για τροποποιημένη Euler και αναλυτική λύση (μεγέθυνση)}
\end{figure}
 \clearpage
 
 \begin{figure}[bh!]
 \centering
\includegraphics[scale = 0.7]{lEuler_c2}
\caption{Μεταβολή της $\psi_2(t)$ για Euler, τροποποιημένη
 Euler και αναλυτική λύση (μικρές διαφορές)}
\end{figure}
 \clearpage
 
\begin{figure}[h!]
 \centering
\includegraphics[scale = 0.45]{lEuler_c2_zoom1}
\caption{Μεταβολή της $\psi_2(t)$ για Euler και αναλυτική λύση (μεγέθυνση)}
\end{figure}

\begin{figure}[h!]
 \centering
\includegraphics[scale = 0.45]{lEuler_c2_zoom2}
\caption{Μεταβολή της $\psi_2(t)$ για τροποποιημένη Euler και αναλυτική λύση (μεγέθυνση)}
\end{figure}
 \clearpage





\begin{appendices}
\section{Αιτιολόγηση Επιλογής Αναπαράστασης Αριθμών με 128 bits}
Αρχικά, στο πρόγραμμα το οποίο αναπτύχθηκε σε γλώσσα C για την εφαρμογή των μεθόδων Euler και τροποποιημένης Euler, χρησιμοποιήθηκαν μεταβλητές τύπου \texttt{long double}. Θέλοντας να επιβεβαιώσουμε τα αποτελέσματα που προέκυψαν, για παράδειγμα στην περίπτωση της εφαρμογής της μεθόδου Euler στο σύστημα με \([f_x, f_y, n_z]^T = [3496, 0, 0]^T\), βρήκαμε την αριθμητική λύση του συστήματος εξισώσεων πρώτης τάξης και στο MATLAB. Χρησιμοποιήθηκε η συνάρτηση \texttt{ode45}, η οποία επιλύει προβλήματα αρχικών τιμών σε συστήματα διαφορικών εξισώσεων πρώτης τάξης, χρησιμοποιώντας τις πιο ακριβείς μεθόδους Runge-Kutta 4$^{ης}$ και 5$^{ης}$ τάξης. Τα αποτελέσματα όμως που πήραμε, είχαν πολύ μεγάλες αποκλίσεις από τα αποτελέσματα του κώδικά μας σε C, όπως φαίνεται από τα σχήματα \ref{l_33}, \ref{l_34}, \ref{l_37} και \ref{l_38}. Προσπαθώντας να προσδιορίσουμε το λάθος:
\begin{enumerate}
	\item Επιβεβαιώσαμε ότι ο κώδικάς μας σε γλώσσα C δίνει σωστά αποτελέσματα σε περιπτώσεις απλών διαφορικών εξισώσεων, όπως στην περίπτωση της εξίσωσης
	\[\dot{y}=-2y^2+yt+t^2, y(t=0)=1\]
τη λύση της οποίας, με τη μέθοδο Euler και τροποποιημένη Euler, πήραμε από τη διαδικτυακή διεύθυνση \texttt{https://math.libretexts.org} (εικόνα \ref{l_32}).
	\item Παρατηρήσαμε ότι στην περίπτωση του συστήματος με \([f_x, f_y, n_z]^T = [3496, 0, 0]^T\), οι τιμές της λύσης της υλοποίησής μας σε C για τη μέθοδο Euler, διαφέρουν απειροελάχιστα (μετά από το 7$^{ο}$ ως 20$^{ο}$ δεκαδικό ψηφίο) από τα αποτελέσματα της λύσης MATLAB-ode45, κατά τα πρώτα βήματα εφαρμογής της μεθόδου, όπως φαίνεται και από τις εικόνες \ref{l_35} και \ref{l_36}. Η διαφορά αυξάνει συσσωρευτικά στα επόμενα βήματα και μετά από 6000 βήματα γίνεται πολύ σημαντική. Επιπλέον, παρατηρήσαμε ότι η εξίσωση για το \(x_1\) είναι πολύ ευαίσθητη, για παράδειγμα, ως προς την αρχική τιμή της μεταβλητής \(x_2\).
	\item Από τα γραφήματα για τα \(x_1\) και \(x_2\) βλέπουμε ότι μετά από κάποιο αρχικό χρονικό διάστημα, το \(x_2\) παίρνει σταθερή τιμή, έστω \(x_2=c=\)σταθερό (κι επομένως και \(\dot{x}_2=0\)) και ότι το \(x_1\) αλλάζει γραμμικά με κλίση \(\dot{x}_1=x_2=c>0\). Επίσης γνωρίζουμε ότι στην περίπτωση \([f_x, f_y, n_z]^T = [3496, 0, 0]^T\) οι μεταβλητές \(\psi_1\) και \(\psi_2\) παραμένουν πάντα μηδενικές. Αντικαθιστώντας λοιπόν \(\dot{x}_2=\psi_1=\psi_2=0\) και \(x_2=c>0\) στην εξίσωση (5), προκύπτει:
	\[0=\frac{f_x-D_xc^2}{m+3m_{\alpha}}\Rightarrow c=\sqrt{\frac{f_x}{D_x}}\]
	και με αντικατάσταση, \(c=0.54350254349\). Μετρώντας πειραματικά την κλίση της καμπύλης \ref{l_33} από το MATLAB, ανάμεσα στα σημεία \((t=400,x_1=176.27141723)\) και \((t=500,x_1=230.55962591)\), προκύπτει 
	\[\frac{230.5596-176.2714}{500-400}=0.5429\]
η οποία συμφωνεί με τη θεωρητική τιμή \(c=0.5435\).
\end{enumerate}
Τα παραπάνω μας οδήγησαν στο συμπέρασμα ότι η διαφορά ανάμεσα στοα αποτελέσματα του κώδικα C και του κώδικα MATLAB δεν οφείλεται σε λανθασμένο κώδικα C, αλλά στην ευαισθησία των διαφορικών εξισώσεων προς ολοκλήρωση και σε συσσωρευμένα αριθμητικά σφάλματα, τα οποία αυξάνονται με το μεγάλο αριθμό βημάτων.

Για το λόγο αυτό, χρησιμοποιήσαμε τη βιβλιοθήκη \texttt{quadmath.h}, η οποία χρησιμοποιώντας τον τύπο \texttt{\textunderscore float128}, επιτρέπει αναπαράσταση floats με ακρίβεια 128 bits. Όλες λοιπόν οι προσομοιώσεις χρησιμοποιούν αναπαράσταση αριθμών με 128 bits για μεγαλύτερη ακρίβεια, και πράγματι τα αποτελέσματα που προκύπτουν με τη χρήση τους ταυτίζονται με αυτά του MATLAB. Για τη σωστή μεταγλώττιση είναι απαραίτητο να εγκατασταθεί η βιβλιοθήκη \texttt{quadmath.h} και να εκτελεστεί η εντολή \texttt{gcc -o <filename> <filename>.c -lquadmath}.
\end{appendices}
\clearpage

\begin{figure}[hb]
 \centering
\includegraphics[scale = 0.5]{MATLAB_Error_x1}
\caption{Λύση της \(x_1\) με ode45}
\label{l_33}
\end{figure}

\begin{figure}[hb]
 \centering
\includegraphics[scale = 0.5]{MATLAB_Error_x2}
\caption{Λύση της \(x_2\) με ode45}
\label{l_34}
\end{figure}
\clearpage

\begin{figure}[hb]
 \centering
\includegraphics[scale = 0.5]{x1_ld}
\caption{Λύση της \(x_1\) με κώδικα C (long double)}
\label{l_37}
\end{figure}

\begin{figure}[hb]
 \centering
\includegraphics[scale = 0.5]{x2_ld}
\caption{Λύση της \(x_2\) με κώδικα C (long double)}
\label{l_38}
\end{figure}
\clearpage

\begin{landscape}
\begin{figure}[hb]
 \centering
\includegraphics[scale = 0.5]{LibrTextsExample}
\caption{Λύση της \(\dot{y}=-2y^2+yt+t^2, y(t=0)=1\) με μεθόδους Euler}
\label{l_32}
\end{figure}
\end{landscape}
\clearpage

\begin{figure}[hb]
 \centering
\includegraphics[scale = 0.8]{EulerErrorMATLAB}
\caption{Λύση των \(x_1, x_2\) με ode45 (πρώτα βήματα)}
\label{l_35}
\end{figure}

\begin{figure}[hb]
 \centering
\includegraphics[scale = 0.6]{long_double_error_c20}
\caption{Λύση των \(x_1, x_2\) με C (πρώτα βήματα)}
\label{l_36}
\end{figure}
\clearpage

\end{document}

